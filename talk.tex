%%%%%%%%%%%%%%%%%%%%%%%%%%%%%%%%%%%%%%%%%%%%%%%%%%%%%%%%%%%%%%%%%%%%%%%%%%%%%%%%
% file:    talk.Rnw
% authors: Peter DeWitt, peter.dewitt@ucdenver.edu %
% Talk to be given at the HPC Launch for Rosliand at UCD.
%
% This file is based on a template made avialble by
% Copyright 2007 by Marco Barisione
% Copyright 2011 by Henricus Bouwmeester
%
% This file may be distributed and/or modified
%
% 1. under the LaTeX Project Public License and/or
% 2. under the GNU Public License.
%%%%%%%%%%%%%%%%%%%%%%%%%%%%%%%%%%%%%%%%%%%%%%%%%%%%%%%%%%%%%%%%%%%%%%%%%%%%%%%%
\documentclass[10pt]{beamer}
%\usefonttheme{serif}
\usepackage[T1]{fontenc}
\usepackage{etex}

%%%%%%%%%%%%%%%%%%%%%%%%%%%%%%%%%%%%%%%%%%%%%%%%%%%%%%%%%%%%%%%%%%%%%%%%%%%%%%%%
% Set up the ucdDenver theme
\newif\ifuseblack
% for black theme, this should be
%     \useblacktrue
% for the white theme, this should be
%     \useblackfalse
\useblacktrue

\usetheme[pageofpages=of,                    % String used between the current page and the
                                             % total page count.
          bullet=circle,                     % Use circles instead of squares for bullets.
          titleline=true,                    % Show a line below the frame title.
          showdate=true,                     % show the date on the title page
          alternativetitlepage=true,         % Use the fancy title page.
          titlepagelogo=./figure/CUAnschutz_c_clr, % Logo for the first page.
          % Logo for the header on first page.
          \ifuseblack
             headerlogo=./figure/pubHealth_tt2_rgb_rv_tp.pdf,
          \else
             headerlogo=./figure/csph_biostatistics-v2.pdf,    % Logo for white background.
          \fi
          watermark=./figure/csph_biostatistics-v2,               % Watermark used in every page.
          watermarkheight=50pt,              % Height of the watermark.
          watermarkheightmult=6,             % The watermark image is 6 times bigger
                                             % than watermarkheight.
          ]{UCDenver}

\ifuseblack
   \usecolortheme{ucdblack}
\else
   \usecolortheme{ucdwhite}
\fi

%%%%%%%%%%%%%%%%%%%%%%%%%%%%%%%%%%%%%%%%%%%%%%%%%%%%%%%%%%%%%%%%%%%%%%%%%%%%%%%%
% Title Page Setup
\author[Peter~E.~DeWitt]{Peter~E.~DeWitt, M.S., Ph.D. Candidate\\{peter.dewitt@ucdenver.edu}\vspace{0.25in}}
\title[]{Statistical Methods Research:\\Running Simulations with R in a High Performance
Computing Environment}
\institute[CSPH, UCD, BIOS]{Colorado School of Public Health\\University of Colorado Denver\\Department of Biostatistics and Informatics}
\date{14 Feburary 2017}
 
%%%%%%%%%%%%%%%%%%%%%%%%%%%%%%%%%%%%%%%%%%%%%%%%%%%%%%%%%%%%%%%%%%%%%%%%%%%%%%%%
\begin{document} 
\watermarkoff

\begin{frame}[t,plain]
  \titlepage
\end{frame}

\begin{frame}[t]
  \frametitle{My Project}
  \begin{itemize}
    \item Working towards a Ph.D. in Biostatistics
    \item[]
    \item Explain the interaction between day-of-cycle, age, and time-to-menopause
      observed in the changes to reproductive hormone profiles expressed during
      the menopausal transition.
    \item[]
    \item Aim 1: Parsimonious B-Spline Regression Model via Control Polygon
      Reduction (CPR)
    \item[]
    \item I've developed a novel method selection approach for enumeration and
      placement of knots for B-splines.  
  \end{itemize}
\end{frame}

\begin{frame}[t]
  \frametitle{Why I Needed A HPC}
  \begin{itemize}
    \item Simulations to show how CPR compared to
      \begin{itemize}
        \item A standard likelihood-based forward-step model selection approach,
        \item A standard likelihood-based backward-step model selection apprach,
        \item P-splines
      \end{itemize}

    \item[]

    \item Start with $L$ internal knots:
      \begin{itemize}
        \item CPR requires $L + 1$ regression fits,
        \item Forward-step requires $L + 1$ regression fits,
        \item Backward-step requires $L(L+1)/2 + 1$ regression fits,
        \item P-splines requires $L + 1$ regressino fits.
      \end{itemize}

    \item[]

    \item For $L = 50$ a total of 1,429 regression models to fit.
      
  \end{itemize}
\end{frame}

\begin{frame}[t]
  \frametitle{Why I Needed A HPC}
  \begin{itemize}
    \item Simulations:
      \begin{itemize}
        \item Four functions.
        \item Modeling under ordinary least squares and under mixed effect models.
        \item Differing sample sizes, model, and subject specific errors.
      \end{itemize}
    \item[]
    \item 144 sets of initial conditions.
    \item[]
    \item Required time to run the $1,429$ regression models for each set of
      inputs: between 00:15 and 03:30.
    \item[]
    \item 1,000 data sets to be generate for each set of initial conditions.  
    \item[]
    \item $205,776,000$ total regression models to fit.
  \end{itemize}
\end{frame}

\begin{frame}
  \frametitle{My HPC Pipline}
  \framesubtitle{Hardware, Software, Disk and RAM Usage}
  \begin{itemize}
    \item Hardware
      \begin{itemize}
        \item A lot of computation cores
      \end{itemize}

    \item Software
      \begin{itemize}
        \item SLURM
        \item R
        \item tar
      \end{itemize}

    \item RAM
      \begin{itemize}
        \item Less than 6 GB per core.
        \item Low enough that the 128GB per Rosliand node was sufficient to
          support 20 parallel simulations per node.
      \end{itemize}

    \item Disk Space
      \begin{itemize}
        \item Less than 2 GB, tarballs total 633 MB.
        \item A lot of small files where generated.
        \item Originally only had 50,000 inode limit on {\tt \$HOME} and
          {\tt \$SCRATCH}, extended to 100,000 inode limit.
      \end{itemize}
  \end{itemize} 
\end{frame}

\begin{frame}[t]
  \frametitle{Difficulties}
  \begin{itemize}
    \item Number of Files.
      \begin{itemize} 
        \item Solution: generate a controlled number of files and place into a
          tarball for each set of initial conditions.
      \end{itemize}

    \item[]

    \item Getting a Job to run to completion.
      \begin{itemize}
        \item Option 1: Request lots of nodes and cores.
          \begin{itemize}
            \item Use GNU parallel to manage the run.
            \item Did not work well for me.
            \item Queue time for resources was high.
            \item Getting a good estimate for the time required to run the sim
              was difficult. (More on this later.)
            \item[]
          \end{itemize}
        \item Option 2: Use SLURM Arrays to request a few resources many times.
          \begin{itemize}
            \item This worked well for me.
            \item Lower queue time.
            \item Leaves more resourse open for others.
          \end{itemize}
      \end{itemize}
  \end{itemize}
\end{frame}

\begin{frame}[t]
  \frametitle{Estimating Time Need For An Iteration}
  \small
  \begin{itemize}
    \item Differences between my desktop and Rosliand
      \begin{tabular}{p{0.75in}ll}
            & Rosliand                                  & My Desktop \\ \hline
        OS  & Red Hat Enterprise 7.2                    & Debian 8.7 (jessie) \\
        CPU & Intel(R) Xeon(R) E5-2680 v3 & Intel(R) Core(TM) i7-3770 \\
        Speed & 2.50GHz & 3.40GHz \\ \hline
      \end{tabular} 

    \item I have configured my desktop to use OpenBLAS for linear algebra.
  \end{itemize}
\end{frame}

\end{document}
%%%%%%%%%%%%%%%%%%%%%%%%%%%%%%%%%%%%%%%%%%%%%%%%%%%%%%%%%%%%%%%%%%%%%%%%%%%%%%%%
% End of file
%%%%%%%%%%%%%%%%%%%%%%%%%%%%%%%%%%%%%%%%%%%%%%%%%%%%%%%%%%%%%%%%%%%%%%%%%%%%%%%%
